
\begin{thanks}
时光荏苒,岁月如梭。读书二十载,此时此刻要说再见了。感谢求学路上遇到的各位老师、同学、朋友和亲人们,感谢你们在我迷茫时为我指引方向,感谢你们在我失落时给予关怀和帮助,感谢你们在我犯错时善意的批评与包容,这一切都是我不断前进的力量,值得铭记一生!

本篇论文是在我的博士生导师陈志明研究员的指导下完成的。感谢陈志明老师这五年来对我学术上的悉心教导和生活上的亲切关怀。陈老师严谨的治学精神,一丝不苟的做事态度和精益求精的工作作风深深感染和激励着我。陈老师以身作则,授人以渔,让我学会独立思考和分析问题的能力。毫不夸张的说,陈老师犹如学术界的一个标杆,他淡泊名利,潜心学问的科研态度值得我一生去学习。

感谢中国科学院计算数学与科学工程研究所给我们提供了良好的学术氛围,优秀的科研平台和舒适的学习环境。所里丰富的科研报告和学术交流让我们随时随地都能聆听数学大师的教诲,与各位优秀的同门讨论学术问题。感谢研究生期间的周爱辉老师、张林波老师、郑文英老师、陈俊清老师、张文生老师、刘晓东老师、张平老师和崔涛老师等各位老师,你们风趣幽默的教学课程和严谨的治学精神让我受益匪浅。


感谢我的同门:黄光辉,向雪霜,梁超,张文龙,李可,周世奇和陈泽材,和你们的交流和讨论让我深受启发。感谢我的同学和朋友:唐仕兵,李凌霄,胡少亮和王兆明等人,和你们一起生活的五年是快乐而充实的。感谢吴继萍老师、尹永华老师、关华老师、邵欣老师、丁汝娟老师、钱莹老师、刘颖老师和陈颖老师等各位中科院数学学院的老师们对我生活和学习上的支持和帮助。

最后要感谢我的父母和兄长,谢谢你们让我这二十多年来在一个幸福美满的家庭中健康快乐地成长。感谢我的父母对我的抚养和教育之恩,是你们默默无闻的付出和毫无条件的支持才能让我在求学路上不断前行,让我变得越来越成熟和懂事。今后的日子里,我将怀着一颗感恩的心,以我最大的努力去回报他们。




\end{thanks}
