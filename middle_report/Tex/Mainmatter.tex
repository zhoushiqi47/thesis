%---------------------------------------------------------------------------%
%->> Main content
%---------------------------------------------------------------------------%

\section{学位论文进展情况,已取得阶段性成果,存在的问题}
通过一段时间的努力研究, 我们针对半空间弹性波散射问题和反散射问题的研究有如下进展及阶段性成果:
  
 利用关于变量 $x_1$ 的 Fourier 变换得到新的易于渐近分析的 Neumann Green 函数的表达式。通过推广传统的用于分析振荡积分的 Van der Corput 引理, 我们推导出在特定区域的 Green 函数的渐近行为。针对半空间弹性波正散射问题, 我们利用经典的极限吸收原理来定义所谓的散射解,并且利用该理论证明了解的适定性。 当嵌入在半空间中的障碍物远离半空间表面边界时,我们描述了半空间散射问题与相应的全空间散射问题两者之间散射解的差距。
	
 针对半空间扩展障碍物重构问题, 我们仅仅利用半空间表面上有限孔径内接收到的单频弹性波数据提出了基于逆时偏移方法的直接成像法。特别地, 我们的成像函数为:
 \ben
 \hat{I}_d(z)=\Im\sum_{q=e_1,e_2}\int_{\Gamma_0^d}\int_{\Gamma_0^d}\,
 [\T_D(x_s,z)^Tq][\T_D(x_r,z)^T\overline{u^s_q(x_r,x_s)}]\,ds(x_r)ds(x_s).
 \een
 这里 $\Ga_0^d=\{x\in\Ga_0: x_1\in (-d,d)\}$, $d>0$, 是半空间表面接收数据的区间,$u_q^s(x_r,x_s)$ 为在 $x_r$ 处接收, 由位于 $x_s$ 处的点源沿着极化方向 $q=e_1, e_2$ 激发的散射数据, $\T_D(x,z)$ 是 Dirichlet Green 函数在 $\Ga_0$ 上 $e_2$ 方向的应力张量。 
 
 我们根据半空间表面上接收数据的孔径大小与嵌入在半空间的障碍物的深度证明了该障碍物重构算法的分辨率,利用点扩散函数的性质来研究分辨率分析,并说明了互相关成像函数的虚部总是在远离障碍物的上边界时衰减到很小。我们用大量的数值实验印证了该直接成像法的有效性和鲁棒性。

存在的问题:

通过数值实验, 我们发现半空间弹性波反散射问题的逆时偏移成像法仅能对障碍的的上边界进行成像, 即障碍物面向于接收面的那部分边界。这与半空间声波情形\cite{RTMhalf_aco}下的数值结果相似。


\section{下一步工作计划,预计答辩时间}
下一步工作计划如下:
\begin{itemize}
	\item 类比声波散射系数来合理定义弹性波散射系数。
	\item 将给出弹性波散射系数的 Kirchhoff 逼近的形式。
	\item 计划利用弹性波散射系数的逼近来说明成像函数只能对障碍物上边界进行成像这一事实。
	\item 针对多个障碍物的情形进行数值实验。
\end{itemize}

预计答辩时间为2019年05月18日。

\section{已取得研究成果列表}

基本撰写完成英文学术论文 《A Direct Imaging Method for Half-Space Inverse Elastic Scattering Problems
》 。%\nocite{*}% 使文献列表显示所有参考文献(包括未引用文献)
%---------------------------------------------------------------------------%
