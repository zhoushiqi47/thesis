\chapter{总结与展望} \label{chap:summary}

本文受地球物理中的勘探模型启发, 针对半空间弹性波反散射问题研究了逆时偏移算法. 在研究反散射问题前, 我们首先系统地研究了半空间弹性波正散射问题.利用 Fourier 变换推导出了两种半空间 Green 函数: Neumann Green 函数和 Dirichlet Green 函数. 进一步, 通过研究振荡积分的衰减性质,给了当 $x\in \Ga_0, \ y\in \R^2_+$ 时, $\N(x,y)$ 和 $\T_D(x,y)$ 随着 $x_2$ 增大的衰减估计.该估计也保证了点扩散函数的定义有意义.然后, 基于点扩散函数和弹性波散射系数的 Kirchhoff 逼近, 我们针对基于逆时偏移的直接成像法给出了严格的数学刻画: 其成像函数在远离散射体边界的时候快速衰减,且只在散射体面向接收面的那部分形成较大的峰值.而且,该分辨率分析不需要高频渐近假设或几何光学近似,对一般的边界条件都成立. 最后,多种数值算例进一步说明该直接成像法的快速有效性、稳定性.

由于弹性波相较于声波的复杂性,我们将从如下几点来叙述有待研究的方向:

1. 关于 Green 函数的研究

由于半空间的Green 函数都是以震荡积分的形式表述,当频率较大时, 我们需要快速算法来计算. 其中,可否将 Green 函数进行渐近级数展开是一个有趣的问题.本文针对 Green 函数衰减性的研究局限在 $\Ga_0$ 上,将来如果可以将研究范围延拓到半空间上甚至是复数域上, 将有助于对半空间弹性波散射问题的 PML 方法的研究.

 本文中, 我们假设半空间背景介质是均匀各项同性的. 但是实际的地质结构是多层非均匀各向异性的. 针对这种更一般的散射问题,首先需要研究半空间弹性波多层介质的 Green 函数的各种性质,例如合适的表达式、表面波的波数等. 

\bigskip
2. 关于接收数据不是全波位移数据的算法

在实际问题中, 数据的相位可能不好获取, 所以在文献 \cite{chen2016direct,chen2017direct,chen2017phaseless} 中 Chen 等针对声波无相位数据,电磁波无相位数据提出了基于逆时偏移的直接成像法. 由于弹性波数据中横波数据和纵波数据耦合在一起, 所以当只接收到混合波的振幅数据时, 很难效仿上述文献的算法来构造成像函数.本文附录\ref{rtm_phaseless} 中, 我们将不加证明地给出针对全空间弹性波反散射问题的无相位数据的直接成像法及相关数值算法.

进一步,针对真实的勘探模型, 一般 $e_2$ 方向的位移数据更容易得到, 于是如何只用接收到的数据 $u^s_q(x_r,x_s)\cdot e_2, \ q=e_1, \ e_2$ 甚至仅用 $u^s_{e_2}(x_r,x_s)\cdot e_2$ 去成像也是个有意义的研究问题.

\bigskip

3. 关于分辨率分析更严谨的数学刻画

需要通过对点扩散函数更为精确的分析, 得出能分辨出的最小距离 $r$, 即
\ben
r=\inf_{t} \{t=\|z-y\| : |[\J(z,y)]_{ii}|=\frac{1}{2}|[\J(z,z)]_{ii}| ,\ i=1,2\}.
\een
另一方面, 在本文中我们利用了 Kirchhoff 逼近来说明重构算法无法对障碍物的阴面进行成像, 所以对该 Kirchhoff 逼近的误差刻画也是非常重要的一个研究问题.进一步, 本文的直接成像函数只是定性的得到障碍物的性质, 如何通过成像函数得出其边界的参数表达,边界条件类型也是有待研究的.
