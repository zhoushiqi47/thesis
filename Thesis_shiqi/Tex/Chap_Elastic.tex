\chapter{半空间中的弹性反散射问题的直接成像方法}\label{chap:Elastic}

\section{Green函数}\label{Green Tensor}
设源点$y\in\R^2_+$, 引入半空间弹性波Neumann零边界格林函数$\N(x,y)$, 对任意向量$q\in\R^2$, 其满足如下方程:
\be
& & \Delta_e [\N(x;y)q] + \omega^2 [\N(x,y)q] = -\mathbf{\delta}_y(x) q \ \ \mbox{in }\R^2_+ , \label{eq_n1} \\
& & \sigma(\N(x,y)q)e_2 = 0 \ \ \mbox{on } \Gamma_0, \label{eq_n2}
\ee
其中(\ref{eq_n2})代表该green函数满足半空间自由边界条件,${\delta}_y(x)$代表位于点y的Dirac源。 由于半空间的特性,我们将利用对$x_1$变量作Fourier变换的方式来推导Green函数,令
\be\label{a1}
\hat \N(\xi,x_2;y_2)= \int_\R\N(x_1,x_2;y) e^{-\i (x_1-y_1)\xi} dx_1,\ \ \forall \xi\in\C,
\ee
记 $\G(x,y)$ \cite{ku63} 为弹性波方程的基本解, 且对其$x_1$变量做Fourier变换后有
$\hat{\G}(\xi,x_2;y_2)=\hat{\G}_s(\xi,x_2;y_2)+\hat{\G}_p(\xi,x_2;y_2)$及
\be
& &\hat{\G}_s(\xi,x_2;y_2)=\frac{\i}{2\omega^2}
\left( \begin{array}{cc}
	\mu_s & -\xi\frac{x_2-y_2}{|x_2-y_2|} \\
	-\xi\frac{x_2-y_2}{|x_2-y_2|} & \frac{\xi^2}{\mu_s}
\end{array} \right)e^{\i\mu_s|x_2-y_2|}, \label{G1}\\
& &\hat{\G}_p(\xi,x_2;y_2)=\frac{\i}{2\omega^2} 
\left( \begin{array}{cc}
	\frac{\xi^2}{\mu_p} & \xi\frac{x_2-y_2}{|x_2-y_2|} \\
	\xi\frac{x_2-y_2}{|x_2-y_2|} & \mu_p
\end{array} \right) e^{\i\mu_p|x_2-y_2|}.\label{G2}
\ee
这里$\mu_\alpha=(k_\alpha^2-\xi^2)^{1/2}$且有$\alpha=s,p$, $k_p=\omega/\sqrt{\lam+2\mu}, k_s=\omega/\sqrt{\mu}$为p波和s波的波数。
为了利用基本解$\G(x,y)$的特性,我们令:
\ben
\N_c(x,y)=\N(x,y)-(\G(x,y)-\G(x,y'))
\een
其中$y'=(y_1,-y_2)$ 为y关于$x_1$轴的镜像点。于是由式(\ref{eq_n1}-\ref{eq_n2}),得$\N_c(x,y)$满足如下方程:
\be
& & \Delta_e [\N_c(x;y)q] + \omega^2 [\N_c(x,y)q] = 0 \ \ \mbox{in }\R^2_+ , \label{eq_n3} \\
& & \sigma(\N_c(x,y)q)e_2 =-\sigma(\G(x,y)-\G(x,y')) \ \ \mbox{on } \Gamma_0, \label{eq_n4}
\ee
\begin{remark}
	在全篇论文中,我们假设对于任意的$z\in \mathbb{C}\backslash\{0\}$, $z^{1/2}$是多值函数$\sqrt{z}$ 的如下解析分支:$\Im(z^{1/2})\geq 0$,这对应于在复平面取右半实轴为割支线。则对于$z=z_1+\mathbf{i}z_2$, $z_1,z_2\in\R$,
	\be \label{convention_1}
	z^{1/2}={\rm sgn}(z_2)\sqrt{\frac{|z|+z_1}{2}}+\i\sqrt{\frac{|z|-z_1}{2}},\ \ \forall z\in\C\backslash\bar{\R}_+.
	\ee
	当$z$位于右半实轴的上沿或是下沿时,取$z^{1/2}$为$\ep\rightarrow0^+$ 时$(z+\i\ep)^{1/2}$ 或是 $(z-\i\ep)^{1/2}$的极限即可。
\end{remark}

通过对式(\ref{eq_n3}-\ref{eq_n4})两边作Fourier变换,我们得到关于变量$x_2$的常系数常微分方程组:
\be
 \mu \frac{d^2(e_1^T\hat \N_c q)}{dx_2^2}+\i(\lambda+\mu)\xi\frac{d(e_2^T\hat \N_c q)}{dx_2}+(\omega^2-(\lambda+2\mu)\xi^2)(e_1^T\hat \N_c q) = 0 \label{eq_n5}\\
 (\lambda+2 \mu)\frac{d^2(e_2^T\hat \N_c q)}{dx_2^2}+\i(\lambda+\mu)\xi\frac{d(e_1^T\hat \N_c q)}{dx_2}+(\omega^2-\mu \xi^2)(e_2^T\hat \N_c q) = 0 \label{eq_n6}
\ee
 由于我们需要$\N(x,y)$为外行波解,因此方程 (\ref{eq_n5})的解为如下两个向量:
\ben
 \left[ \begin{array}{cc} \i\mu_s \\ -\i\xi \end{array} \right]e^{\i\mu_s x_2} \ , \ \ \ \ \ \left[ \begin{array}{cc} \i\xi \\ \i\mu_p \end{array} \right]e^{\i\mu_p x_2}
\een
的线性组合。 利用边界条件(\ref{eq_n6})及待定系数法,我们得到:
\be\label{NGT}
\hspace{-2cm}\hat \N_c(\xi,x_2;y_2) =  \frac{\i}{\omega^2\delta(\xi)}\sum_{\alpha,\beta=p,s}\mathbb{A}_{\al\beta}(\xi)e^{\i(\mu_\al x_2+\mu_{\beta} y_2)}, 
\ee
其中 $\varphi(\xi)=k_s^2-2\xi^2$, $\delta(\xi)=\varphi(\xi)^2+4\xi^2\mu_s\mu_p $(Rayleigh方程\cite{achenbach1980}), 以及
\ben
&&{\mathbb{A}_{ss}(\xi)} =
\left( \begin{array}{ll}
	\varphi^2\mu_s & -4\xi^3\mu_s\mu_p \\
	-\xi\varphi^2  & 4\xi^4\mu_p
\end{array} \right),\ \ 
{\mathbb{A}_{sp}(\xi)} =
\left( \begin{array}{ll}
	2\xi^2\varphi\mu_s & -2\xi\varphi\mu_s\mu_p \\
	-2\xi^3\varphi  & 2\xi^2\varphi\mu_p
\end{array} \right),\\
&&
{\mathbb{A}_{ps}(\xi)} =
\left( \begin{array}{ll}
	2\xi^2\varphi\mu_s & 2\xi^3\varphi \\
	2\xi\varphi\mu_s\mu_p  & 2\xi^2\varphi\mu_p
\end{array} \right),\ \ 
{\mathbb{A}_{pp}(\xi)} =
\left( \begin{array}{ll}
	4\xi^4\mu_s & \xi\varphi^2 \\
	4\xi^3\mu_s\mu_p  & \varphi^2\mu_p
\end{array} \right).
\een
按照惯例,原本我们只要对$\hat{\N}(\xi,x_2;y_2)$进行Fourier逆变换就可以得到所需要的Neumann Green函数. 然而,如下面的引理所述, 函数$\delta(\xi)$在实轴上存在零点\cite{achenbach1980, Harris2001Linear},此时我们并不可以对其直接进行Fourier逆变换.
\begin{lem} \label{rayleigh}
	 Rayleigh方程 $\delta(\xi) = 0$在复平面$\C$中有且仅有两个根且记为 $\pm k_R$, 其中$k_R$满足$k_R>k_s$。
\end{lem}

\debproof
 由前文注记中的(\ref{convention_1}), 易得 $\delta(\xi)$ 的割支线为 $C_l=\{\xi=\xi_1+\i\xi_2\in\C: \xi_1\in [-k_s,-k_p],\xi_2=0\}$ 和 
$C_r=\{\xi=\xi_1+\i\xi_2\in\C: \xi_1\in [k_p,k_s],\xi_2=0\}$. 于是$\delta(\xi)$在除 $C_l$ 和 $C_r$ 以外的区域解析。 而在割支线上,$\delta(\xi)$ 可表示成: 
\ben
\delta(\xi)=(k_s^2-2\xi^2)^2+\i\,[4\xi^2(k_s^2-\xi^2)^{1/2}(\xi^2-k_p^2)^{1/2}], \ \ \forall \xi\in C_l\cup C_r.
\een
显然, $\de(\xi)$ 在 $C_l\cup C_r$ 上没有零点。 又因为 $\de(\pm k_s)>0$ , $\de(\pm\infty)<0$ ,由函数的连续性得 $\de(\xi)$ 在区间 $(-\infty,-k_s)\cup(k_s,\infty)$ 上至少存在两个零点, 且由于其对称性,可以记为 $\pm k_R$。 下面, 我们将 $C_l, \ C_r$ 的上下沿分别记为 $C_l^\pm, \ C_r^\pm$。

接下去, 利用幅角原理\cite{Ahlfors1979Complex}可以说明 $\delta(\xi)$ 在整个复平面只存在两个零点。 令 $\Ga_R$ 为半径 $R$ 充分大的圆. 我们考虑 $\mathcal D$ 是被周线 $\Ga_R$, $\Ga_l$ 以及 $\Ga_r$ 包围的区域。 其中 $\Ga_l$ 代表沿着 $C_l^+$ 从 $-k_s$ 到 $-k_p$  及然后沿着 $C_l^-$ 从 $-k_p$ 到 $-k_s$ ; 相应地, $\Ga_r$ 代表 沿着 $C_r^+$ 从 $k_p$ 到 $k_s$ 及然后 沿着 $C_r^-$ 从 $k_s$ 到 $k_p$ 。 因为 $\delta(\xi)$ 在整个整个复平面上没有极点,  我们可以通过幅角原理来计算其在区域
 $\mathcal D$ 中的零点个数 $Z$ :
\be\label{zero}
Z=\frac{1}{2\pi\i}\int_C \frac{\delta'(\xi)}{\delta(\xi)}d\xi.
\ee
由式子(\ref{convention_1})中的定义,我们可以得出当$\xi\in C_r^\pm$时 $\de(\xi)=\de^\pm(\xi)$, 其中
\ben
\de^\pm(\xi)=(k_s^2-2\xi^2)^2\mp\i\,[4\xi^2(k_s^2-\xi^2)^{1/2}(\xi^2-k_p^2)^{1/2}\,]:=f_1(\xi)\mp\i f_2(\xi).
\een
于是可以有如下计算
\ben
\int_{\Ga_r} \frac{\delta'(\xi)}{\delta(\xi)}d\xi&=&\int_{k_p}^{k_s}\left(\frac{{\delta}_{+}' (\xi)}{\delta_{+}(\xi)}-\frac{{\delta}_{-}' (\xi)}{\delta_{-}(\xi)}\right) d\xi\\
&=&2\i\int_{k_p}^{k_s}\frac{f_1'(\xi) f_2(\xi)-f_1(\xi) f_2'(\xi)}{f_1^2(\xi)+ f_2^2(\xi)} d\xi\\
&=&-2\i\arctan \frac{f_2(\xi)}{f_1(\xi)}\Bigg|^{k_s}_{k_p}=0.
\een
相似地, 在 $\xi\in C_r^\pm$ 时也有 $\int_{\Ga_l}\frac{\delta'(\xi)}{\delta(\xi)}d\xi=0$ 。 此外, 当 $|\xi|$ 足够大, 容易得到 $\de(\xi)$ 的渐近形式 $\delta(\xi)=-2(k_p^2+3k_s^2)\xi^2+O(1)$ 及 $\delta’(\xi)=-4(k_p^2+3k_s^2)\xi+O(1)$ 。 于是当 $R\gg 1$ ,可以计算得到
$\int_{\Ga_R} \frac{\delta'(\xi)}{\delta(\xi)}d\xi=4\pi\i$ 。
综上所述, 我们得出 $Z=2$ 。 于是该引理得到证明。
\finproof

为了解决这一问题,我们假设半空间的介质是耗散的,然后研究其相应的Green函数,最后通过极限吸收原理得到$\N(x,y)$。
记 $\mathbb{N}_{\omega(1+\i\ep)}(x,y)$ 为满足将式子(\ref{eq_n1})中将实圆频率$\omega$ 替换为复圆频率$\om(1+\i\ep)$后相应方程的Green函数。 同样的, 对$\mathbb{N}_{\omega(1+\i\ep)}(x,y)$关于$x_2$变量的Fourier变换,得到$\hat\N_{\omega(1+\i\ep)}(\xi,x_2;y_2)$,且其表达式与将(\ref{NGT})中将$k_s, k_p$替换为
$k_s(1+\i\ep), k_p(1+\i\ep)$后相应的式子一致。 下面的引理告诉我们,$\hat\N_{\omega(1+\i\ep)}(\xi,x_2;y_2)$的零点所在何处。

\begin{lem}\label{complex_rayleigh}
	令$\delta_{\om(1+\i\ep)}(\xi)$为将$\delta(\xi)$中的$k_p,k_s$替换成$k_s(1+\i\ep), k_p(1+\i\ep)$后相应的复Rayleigh方程。 那么, $\delta_{\om(1+\i\ep)}(\xi)$在$\C\bks\bar{\Om}$中有且仅有两个根且为$\pm k_R(1+\i\ep)$。 其中集合$\Om$为
	\be\label{set:Om}
	\Omega := \{\xi_1+\i\xi_2 \in \mathbb{C} \ | \ k_p(1+\i\ep)<\xi_1\xi_2<k_s(1+\i\ep) \ , \  \ \xi_2>\xi_1(1+\i\ep)\}
	\ee
\end{lem}

 由引理\ref{complex_rayleigh}得,$\hat\N_{\omega(1+\i\ep)}(\xi,x_2;y_2)$在实轴上没有极点, 可以对其直接进行逆Fourier变换。 于是, Neumann Green函数 $\mathbb{N}(x,y)$ 可以利用极限吸收原理得到,即为
\be\label{NGT1}
\hspace{-1.5cm}\N(x,y)=\lim_{\ep\to 0^+} \N_{\om(1+\i\ep)}(x,y)=\lim_{\ep\to 0^+}\frac{1}{2\pi}\int_\R\hat \N_{\om(1+\i\ep)}(\xi,x_2;y_2) e^{\i(x_1-y_1)\xi} d\xi.
\ee

现在, 我们已经得到 Neumann Green函数的具体表达形式了。 但是,式子(\ref{NGT1})中这种极限形式并不利于我们分析该函数的具体性质,特别是其无穷远处的衰减阶数。 为了便于得到更加简洁的表达形式, 我们引入下面这个关于柯西主值(cf. e.g. \cite[Chapter 4, Theorem 5]{Kuroda})的引理
\begin{lem}\label{cauchy_pv}
	令 $a,b\in\R,\  a<b$, 且 $t_0\in (a,b)$. 如果 $\gamma$ 在$[a,b]$上 H\"older 连续, 即存在常数 $\alpha\in (0,1]$ 及 $C>0$ 对于任意 $s,t\in [a,b]$, $|\gamma(s)-\gamma(t)|\le C|s-t|^\alpha$, 于是有
	\ben
	\lim_{z\to t_0,\pm\Im z>0}\int^b_a\frac{\gamma(t)}{t-z}dt={\rm p.v.}\int^b_a\frac{\gamma(t)}{t-t_0}dt\pm\pi\i\ga(t_0),
	\een
	其中 ${\rm p.v.}\int^b_a$ 表示积分的Cauchy主值。
\end{lem}
\section{正散射问题的适定性}

