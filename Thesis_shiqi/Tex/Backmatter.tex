\chapter{作者简历}



\section*{基本情况}


周世奇,浙江省绍兴人,中国科学院数学与系统科学研究院,在读博士研究生.

\section*{研究方向}

反散射问题, 弹性波方程, 逆时偏移算法.


\section*{已发表的学术论文:}

[1] Z. Chen, S. Zhou. A Direct Imaging Method for Half-Space Inverse
Elastic Scattering Problems, Inverse Problems, Accepted.

\section*{联系方式}

通讯地址:北京市海淀区中关村东路55号,中科院数学与系统科学研究院,

邮编:100190,

E-mail: shiqizhou@lsec.cc.ac.cn ,

电话:18518316647.

\chapter[致谢]{致\quad 谢}\chaptermark{致\quad 谢}% syntax: \chapter[目录]{标题}\chaptermark{页眉}
\thispagestyle{noheaderstyle}% 如果需要移除当前页的页眉
博士论文已撰写完成,二十一年的求学生活随之进入尾声.回首往事,我心中充满感激.%\pagestyle{noheaderstyle}% 如果需要移除整章的页眉

首先感谢我的导师陈志明研究员在这五年来对我的悉心指导与关怀.感谢陈老师一次又一次对我的教诲,教给我生活的道理,分享给我人生的经验.陈老师在这五年来提供了宽松的研究环境,让我可以研究自己喜欢的问题.在陈老师细心的指导下,我领略到计算数学之美,领略到数学家思考问题的方式.陈老师淡泊名利,潜心学问的科研态度让我见识到了大师的风范,值得我一生去学习.

感谢中国科学院计算数学与科学工程研究所给我们提供了良好的学术氛围,
优秀的科研平台和舒适的学习环境.所里丰富的科研报告和学术交流让我们随时
随地都能聆听数学大师的教诲.感谢研究生期
间的周爱辉老师,张林波老师,郑伟英老师,陈俊清老师,张文生老师,毛士鹏老师,刘晓东
老师等各位老师,你们风趣幽默的教学课程和严谨的治学
精神让我受益匪浅.


感谢我的同门:黄光辉博士,向雪霜博士,张文龙博士,李可博士,方少峰博士和陈泽材师弟,与
你们的交流和讨论让我深受启发.感谢我的同学和朋友:黄猛,刘化庆, 王义舒,崔建波,高斌,余翌帆,李英哲,陈伟坤,赵亮,何睿,王乔,宋荣荣,张宁,王亚辉,巫文婷,纪楠等人,和你们一起生活的五年是快乐而充实的,点点滴滴都值得回忆.感谢吴继萍老师,尹
永华老师,关华老师,邵欣老师,丁汝娟老师,钱莹老师,刘颖老师
等各位中科院数学学院的老师们对我生活和学习上的支持和帮助.

特别感谢我的初中老师黄志勇先生.回首漫漫求学路,我取得的每一点进步都深深地植根于您十三年前的教诲.

最后,我要感谢我的家人.感谢我深爱的父亲母亲,做你们的儿子是我此生最幸运的事情,感谢你们对我的抚养和教育之恩,是你们默默无闻
的付出和毫无条件的支持才能让我在求学路上不断前行.感谢我的岳父岳母,感谢你们这几年来对我生活上的照顾.感谢我的妻子唐凯莉女士,在我失落的时候,你是我内心最大的慰藉,在我遭受挫折的时候,想到你我总能找回前进的勇气,感谢你的陪伴.

谨以此文献给我们的青春.

\cleardoublepage[plain]% 让文档总是结束于偶数页,可根据需要设定页眉页脚样式,如 [noheaderstyle]

