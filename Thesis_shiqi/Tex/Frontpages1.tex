%---------------------------------------------------------------------------%
%->> 封面信息及生成
%---------------------------------------------------------------------------%
%-
%-> 中文封面信息
%-
\confidential{}% 密级:只有涉密论文才填写
\schoollogo{scale=0.095}{ucas_logo}% 校徽
\title{ 半空间中的弹性波反散射问题}% 论文中文题目
\author{}% 论文作者
\advisor{}% 指导教师:姓名 专业技术职务 工作单位
\advisorsec{}% 指导老师附加信息 或 第二指导老师信息
\degree{博士}% 学位:学士、硕士、博士
\degreetype{理学}% 学位类别:理学、工学、工程、医学等
\major{计算数学}% 二级学科专业名称
\institute{中国科学院~数学与系统科学研究院}% 院系名称
\chinesedate{2019~年~6~月}% 毕业日期:夏季为6月、冬季为12月
%-
%-> 英文封面信息
%-
\englishtitle{ Inverse Elastic Scattering Problems in the Half Space}% 论文英文题目
\englishauthor{}% 论文作者
\englishadvisor{Supervisor: }% 指导教师
\englishdegree{Doctor}% 学位:Bachelor, Master, Doctor。封面格式将根据英文学位名称自动切换,请确保拼写准确无误
\englishdegreetype{Natural Science}% 学位类别:Philosophy, Natural Science, Engineering, Economics, Agriculture 等
\englishthesistype{thesis}% 论文类型: thesis, dissertation
\englishmajor{Computational Mathematics}% 二级学科专业名称
\englishinstitute{Academy of Mathematics and Systems Science \\
	Chinese Academy of Sciences}% 院系名称
\englishdate{June, 2019}% 毕业日期:夏季为June、冬季为December
%-
%-> 生成封面
%-
\maketitle% 生成中文封面
\makeenglishtitle% 生成英文封面
%-
%-> 作者声明
%-
\makedeclaration% 生成声明页
%-
%-> 中文摘要
%-
\chapter*{摘\quad 要}\chaptermark{摘\quad 要}% 摘要标题
\setcounter{page}{1}% 开始页码
\pagenumbering{Roman}% 页码符号

本文主要研究了半空间时谐弹性波反散射问题的逆时偏移算法及其数学理论分析. 近几十年来,  该问题在多种学科领域内受到广泛关注,  其中包括无损探测领域,医学领域以及地震波勘探领域.  一方面, 传统的地震波数据处理都是基于声波方程的, 而弹性介质中允许存在横波和纵波. 此外, 半空间自由边界条件的存在导致出现了只沿着半空间表面传播的 Rayleigh 表面波. 另一方面, 从反问题的角度看,  散射问题的逆算子具有高度的不适定性及非线性性. 因此, 上述所有的困难都给本文的研究带来挑战.  本文的研究涉及以下两大方面:

\begin{itemize}
	\item  利用关于变量 $x_1$ 的 Fourier 变换得到新的易于渐近分析的 Neumann Green 函数及 Dirichlet Green 函数的表达式. 通过推广传统的用于分析振荡积分的 Van der Corput 引理,  我们推导出在特定区域的 Green 函数的渐近行为. 针对半空间弹性波正散射问题,  我们利用经典的极限吸收原理来定义所谓的散射解, 并且利用该理论证明了解的适定性.  当嵌入在半空间中的障碍物远离半空间表面边界时, 我们描述了半空间散射问题与相应的全空间散射问题两者之间散射解的差距. 
	
	\item 针对半空间扩展障碍物重构问题,  我们提出了一种基于逆时偏移思想的直接成像法, 该算法仅需要利用半空间表面上有限孔径内接收到的单频弹性波数据. 我们根据半空间表面上接收数据的孔径大小与嵌入在半空间的障碍物的深度证明了该障碍物重构算法的分辨率. 在分辨率分析中利用点扩散函数的性质, 并说明了互相关成像函数的虚部总是在障碍物的上边界上达到峰值. 我们用大量的数值实验印证了该直接成像法的有效性和鲁棒性. 
\end{itemize}

\keywords{半空间, 弹性波方程, 反散射问题, 逆时偏移}% 中文关键词
%-
%-> 英文摘要
%-
\chapter*{Abstract}\chaptermark{Abstract}% 摘要标题

In this thesis, we focus on developing reverse time migration method for time harmonic elastic wave inverse scattering problems in the half space and establishing the related mathematical theory. In recent decades, this problems have considerable interests in diverse application fields including non-destructive testing, medical imaging, and especially seismic exploration.
Seismic processing
 usually is based on acoustic equations, but elastic materials allow for both compressional and shear wave propagation. Moreover, the free boundary condition induces the propagation of a Rayleigh surface wave guided by the unbounded flat surface of the half space. From the perspective of inverse problems, the inverse opertor of the scattering problem is improperly posed and inherently nonlinear. Therefore, the challenges of this thesis include all above difficulties. The thesis consists of the following issues:
\begin{itemize}
	\item By using Fourier transform with respect to the horizontal variable $x_1$, we drive a new expression for the Neumann Green Tensor and Dirichlet Green Tensor. Based on the extention of the classic Van der Corput lemma, we show the asymptotic behavior of Green Tensor in a specific area. For the forward elastic scattering problems in the half space, we take the method of limiting absorption principle to define  sacttering solution and obtain its well-posedness. Besides, we discribe the difference between the half-space scattering solution and the full space scattering solution  when the scatter is far away from the boudary of the half space.
	
	
	\item We propose a direct imaging method based on the reverse time migration to reconstruct extended
	obstacles in the half space with finite aperture elastic scattering data at a fixed
	frequency. We also prove the resolution of the reconstruction method in terms of the
	aperture and the depth of the obstacle embedded in the half space. The resolution
	analysis is studied by virtue of the point spread function and implies that the imaginary 
	part of the cross-correlation imaging function
	always peaks on the upper boundary of the obstacle. Numerical examples
	are included to illustrate the effectiveness of the method. 
\end{itemize}

\englishkeywords{half space, elastic equations, inverse scattering problems, reverse time migration}% 英文关键词
%---------------------------------------------------------------------------%
