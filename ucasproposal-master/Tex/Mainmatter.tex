%---------------------------------------------------------------------------%
%->> Main content
%---------------------------------------------------------------------------%
\section{选题的背景及意义}
近几十年来, 弹性波散射问题及其反问题在工程领域和数学领域都得到了广泛研究\cite{landau}, 其中包括无损探测领域、医学领域以及地震波勘探领域。

散射理论研究在二十世纪的数学物理学界占据了非常重要的地位。通俗地讲, 弹性波的散射问题, 就是研究当入射波(地震源、可控震源)从一种弹性介质进入另一种弹性介质或是碰到障碍物 (散射体) 时产生的效应。 与声波、电磁波散射问题提法类似, 弹性波散射问题只是把控制方程换成了 Navier 方程。特别地, 如果我们把弹性波总场 $u(x)$ 看作是入射波 $u^i(x)$ 和散射波 $u^s(x)$的和,那么正散射问题就是根据入射波 $u^i(x)$,障碍物的物理特性以及弹性波方程去决定散射波 $u^s(x)$。 而在研究反散射问题前, 对正散射问题有一个清晰的理解是必不可少的。

弹性波反散射问题是利用接收到的弹性波散射数据去决定障碍物的位置、形状、大小。 一方面,传统的地震波数据处理都是基于声波方程的,而弹性介质中允许存在横波和纵波。此外,半空间自由边界条件的存在导致出现了只沿着半空间表面传播的 Rayleigh 表面波。另一方面,从反问题的角度看, 散射问题的逆算子具有高度的不适定性及非线性性。因此,上述所有的困难都给半空间弹性波反散射问题的研究带来挑战及吸引力。 而在地震波勘探模型中,弹性波是在地表以下传播的,这其实就是一个半空间弹性波散射问题。于是, 联系实际模型,从数学的角度来研究半空间弹性波散射及反散射问题不仅可以建立完善的理论框架, 而且能更加本质的理解模型的物理意义。


\section{国内外本学科领域的发展现状与趋势}
由于反问题是高度线性不适定问题 \cite{hadamard1923lectures}, 换言之, 如果我们测量到的数据不准确或是有一个微小的扰动, 都会可能导致相对应的障碍物带来巨大的误差。因此对反问题做适当形式的转化, 是求解反问题的重要方法。类似于声波、电磁波, 弹性波反散射问题的方法主要也分为迭代法和直接成像法两种。 

迭代法是一种优化手段, 具体是将观察数据与初值参数构成目标函数, 然后将目标函数最小化的过程。 抽象为如下表达式:
\ben
\min_{\mathbf{m}} \ dist(\mathbf{d}^{obs},\mathbf{{L}}\mathbf{m})
\een
其中 $\mathbf m$ 表示障碍物边界参数或是非均匀介质参数, $\mathbf{d}^{obs}$ 表示观测数据, $\mathbf{{L}}$ 表示弹性波方程的解算子, $dist(\cdot,\cdot)$ 表示某种距离。
对于全空间弹性波障碍物散射问题, Li \cite{li2016inverse} 提出了利用多频散射数据,基于区域导数的牛顿迭代法。 针对半空间非均匀弹性介质, Mora等人 \cite{mora1987nonlinear} 提出了最小二乘法反演介质参数的方法。 这些数值迭代方法的优势就是可以定量的得到障碍物的边界或是介质参数信息, 但是要求一定的先验信息, 例如边界条件等。 由于每一次迭代都需要解一次弹性波正问题, 这无疑是相当巨大的计算量。 而且,这种优化问题都是非凸的, 导致初值的选择, 迭代收敛性的证明都是比较困难的问题。 

直接成像法的基本思想是构造一个指示性成像函数,代入观测数据后, 该函数值在远离障碍物边界时逐渐衰减; 在靠近边界时, 函数值趋于峰值。针对二维全空间弹性波反障碍物散射问题, Arens \cite{arens2001linear} 将线性采样法 (Linear Sampling Method) 从声波情形推广了到了弹性波情形。  Alves 和 Kress \cite{alves2002far}, Arens \cite{arens2001linear}, Charalambopoulos \cite{charalambopoulos2006factorization}, 以及  Hu, Kirsch \cite{hu2012some} 等发展了弹性波情形下的分解法 (Factorization Method) 。 然而, 无论是线性采样法还是分解法, 至今都不能应用于弹性波反散射问题的近场数据。由于, 我们没有办法将接收到的近场混合波数据分解成 p 波和 s 波, 除非我们能得到接收点附近领域内的所有数据。 Guzina 等
人 \cite{gintides2012identification} 利用拓扑导数方法对弹性波非均匀介质逆散射问题进行重构。 Chen 和 Huang \cite{ela_reverse} 本文针对全空间弹性介质扩展障碍物成像问题, 提出了单频加权弹性波逆时偏移方法,给出了弹性波逆时偏移方法的分辨率分析,该理论结果表明成像函数均为正值。

目前,在数学物理领域中, 对弹性波反散射问题的讨论主要集中上述的全空间反障碍物散射问题,以及粗糙曲面反散射问题\cite{hu2016factorization,li2016near},至今极少有数学方面的文献讨论半空间障碍物模型的。 然而,在地球物理领域中, 应用于地下半空间成像的算法已近发展了几十年, 其中逆时偏移算法是近二十年来最受关注的成像算法。

逆时偏移 (Reverse Time Migration) 是源于勘探地球物理领域的一种叠前深度偏移方法。 在逆时偏移流行之前, 单程波方程偏移法\cite{claerbout1972downward,gazdag1978wave} 一直是偏移研究的主要课题。但是, 为了将波动方程分解成单程的上行波方程和下行波方程,需要引入平方根算子\cite{zhanggq1993,Zhang2007}。 但是, 平方根算子是一个拟微分算子,在数值计算时, 需要用一系列的积分、微分算子来逼近,这给单程波方程的计算带来非常大的困难。于是,基于全波方程的逆时偏移法在 1983年被 Whitmore \cite{whitmore1983iterative}, Baysal \cite{baysal1983reverse} 和 Mcmechan  \cite{mcmechan1983migration} 先后提出。 早期, 由于计算资源的匮乏, 工程师们把接收到的弹性波数据利用声波方程来进行逆时偏移 \cite{zhang2009,Zhang08,bleistein2013mathematics,claerbout1985imaging,berkhout2012seismic},但是由于弹性波是包含两种不同波数的 s 波和 p 波的耦合波, 两种波携带着不同的位移信息\cite{yan2008isotropic}。 因此, 发展利用弹性波全波方程的逆时偏移方法是必然的。Chang 和 McMechan \cite{chang1986reverse} 利用弹性波方程将接收到的弹性波数据时逆地外推到地表下, 然后使用激励时间成像条件 (Excitation time), 而 Hokstad\cite{hokstad1998elastic} 利用 {Lam\'{e}} 势方法作为成像条件。 总之, 这两种成像条件都是互相关成像条件的特殊情形\cite{yan2008isotropic}。此外, 由于弹性波包含 s 波和 p 波,地球物理学家发展出了两种基于 Helmholz 分解的逆时偏移算法 \cite{yan2008isotropic,sun2001scalar,denli2008elastic,chung2012implementation}。 

第一种方法是将接收到的弹性波数据,反传到接收面附近的浅层区域, 利用 Helmholtz 分解将耦合波场分解成 s 波向量势和 p 波标量势 \cite{etgen1988prestacked,zhe1997prestack}。随后, 利用相应的 s 波波数与 p 波波数的声波方程将分解后的波正传回接收面。随后利用传统的声波逆时偏移算法, 对分解完的数据进行成像。

第二种方法是先用弹性波全波方程将在表面接收到的位移数据反传到地下,然后将反传后的弹性波与点源发出的弹性波都进行波场分解, 最后对分解后的入射波与反传波作互相关\cite{dellinger1990wave}。

对于逆时偏移方法的理论分析, 最早是由 Beylkin \cite{beylkin1984inversion,beylkin1985imaging,beylkin1990linearized} 基于广义 Radon 变换, 采用高
频渐近假设或者几何光学近似给出的渐近分析。 但是, 在实际的工程情况下, 这些假设的条件一般都不能满足。近几年来, Chen 等\cite{chen2013reverse_acou,chen2013reverse_elec,thesis_guanghui} 针对全空间中声波和电磁波反障碍物散射问题, 研究了相应的单频逆时偏移方法来重构障碍物。 他们基于  Helmholtz-Kirchhoff 等式, 在不需要高频假设或几何光学近似的前提下,对算法的分辨率做出来严格分析, 并证明了该成像函数恒为正函数, 从而保证了改数值方法的稳定性。在文献 \cite{chen2015reverse_planar} 中, Chen 等针对平行平板声波波导反散射问题,提出了基于逆时偏移算法的直接成像法。该文章中, 作者提出了 广义的 Helmholtz-Kirchhoff 等式,并由此给出了成像分辨率的理论结果。进一步, 文献\cite{RTMhalf_aco}中作者考虑声波半空间反散射问题的逆时偏移方法,并提出了全新的点扩散函数, 从而给出了分辨率的理论分析。 特别地, 该文章给了孔径选取的标准, 以及说明了半空间情形下只能对障碍物面向接收面的那部分进行成像。

由于在实际的工程应用中, 获取观测数据的强度或振幅 (无相位数据) 比获取该观测数据的相位信息要容易很多。由此, Chen等 \cite{chen2016direct} 基于逆时偏移算法,针对全空间声波反散射问题, 提出了无相位成像算法。 此外, 当障碍物远离发射面和接收面时, 证明了该无相位成像算法与原本的逆时偏移成像方法近似等价。



\section{课题主要研究内容、预期目标}
 传统的地震波数据处理都是基于声波方程的,而弹性介质中允许存在横波和纵波。此外,半空间自由边界条件的存在导致出现了只沿着半空间表面传播的 Rayleigh 表面波。而在地震波勘探模型中,弹性波是在地表以下传播的,这其实就是一个半空间弹性波散射问题。正是半空间自由表面的存在, 导致在半空间传播的波中出现了除横波和纵波两种体波以外的表面波,这使得半空间散射问题与全空间散射问题的本质不同。于是,我们将联系实际模型,基于弹性波全波方程针对半空间散射问题和反散射问题加以研究,特别是考虑如何从数值上构造一个高效稳定的算法来重构障碍物。

我们的预期目标是针对半空间弹性波散射问题, 给出散射解的定义以及解的适定性。针对半空间弹性波反散射问题,我们准备基于逆时偏移思想提出一个重构障碍物的直接成像法,且该算法不需要事先知道障碍物是否可穿透以及其不可穿透时边界条件的先验信息。 最后,我们将用严格的数学理论来针对我们所提出的重构算法的进行分辨率分析,并且用大量数值算例来加以印证。

\section{拟采用的研究方法、技术路线、实验方案及其可行性分析}

首先针对半空间弹性波正散射问题, 我们将利用经典的极限吸收原理来定义所谓的散射解,而后研究是否可以得出解的适定性。 因此, 我们需要先深入理解极限吸收原理的经典数学理论, 去构造适用于半空间弹性波散射问题的解的函数空间。

而后, 针对半空间障碍物反散射问题, 我们首先考虑当障碍物是点源时的特殊情形。 于是,需要研究点源即相应边界条件下的半空间 Green 函数的性质。 由于半空间中 $x_1$ 变量是填充整个实数域 $\R$ 的, 拟采用 Fourier 变换得到 Green 函数的表达式。通过研究振荡积分的理论,来估计 Green 函数的渐近行为。受文献\cite{RTMhalf_aco}中针对半空间声波反散射问题的研究, 而后定义针对点源成像函数为弹性波点扩散函数, 且需要研究该点扩散函数的性质。

依据逆时偏移的思想, 我们打算仅仅利用半空间表面上有限孔径内接收到的单频弹性波数据提出了基于逆时偏移方法的直接成像法。 有了以上研究基础,我们将可以针对半空间扩展障碍物重构问题做理论分析,依据点扩散函数的性质来研究基于弹性波逆时偏移方法的直接成像法的分辨率。最后, 我们用大量的数值实验来检验该成像函数是否有效。
\section{已有科研基础与所需的科研条件}
通过平时课程的学习及文献的阅读, 已经具备了扎实的数学理论基础。 针对将要研究的半空间弹性波散射问题和反散射问题已经有了初步的认识。具备扎实的 Matlab, C语言的编程能力, 能对相关数值算法进行实现。


\section{研究工作计划与进度安排}

前期准备先推导半空间中在表面处满足 Neumann 边界条件的弹性波 Green 函数的表达式。 然后,由于点扩散函数的需要, 推导半空间中在表面处满足 Dirichlet 边界条件的弹性波 Green 函数及其应力张量的表达式。 有了上述表达式, 可以计划在数值上实现嵌入在半空间中点源的成像, 通过成像结果观察点扩散函数是否具备峰值 。 如果数值实验成果符合预期, 再试图用数学理论证明成像结果中函数的性质,于是涉及到 Green 函数性质的研究。

中期研究半空间弹性波正文题散射解的定义及解的适定性。 而后, 通过有限元方法或是边界积分法来计算数值解,为反问题成像算法的数值实验提供数据。而后, 基于逆时偏移思想来构造反散射问题的直接成像法。 有了直接成像法, 我们可以通过其数值格式来进行实验,来验证该成像函数是否真实有效。

后期可以利用正散射问题散射解的适定性及受数值实验中成像结果的启发来初步进行重构算法的分辨率分析。最后试图利用严格的数学理论刻画成像函数的各种性质。



%\nocite{*}% 使文献列表显示所有参考文献(包括未引用文献)
%---------------------------------------------------------------------------%
